\chapter{Quantum}
\label{chap:Quantum}

\section{State Space}
The bit is the fundamental unit of information for classical information processing. In quantum information processing, the corresponding unit is the qubit, which is described mathematically by a vector of length one in a two-dimensional complex vector space. We use notation from physics to denote vectors that represent quantum states, enclosing vectors in a ket.

\section{Unitary evolution and circuits}
Basic evolutions of a quantum system are described by linear operations that preserve the norm; formally, these operations can be expressed as unitary complex matrices

Quantum algorithms are commonly described as circuits (rather than by quantum Turing matrix machines) consisting of basic quantum gates from a universal set

\section{measurement}
In addition to unitary evolution, we specify an operation called measurement, which, in the simplest case, takes a qubit and outputs a classical bit

We further specify that, after the process of measurement, the quantum system collapses to the measured outcome. Thus, the quantum state is disturbed and it becomes classical: any further measurements have a deterministic outcome. We have described measurement with respect to the standard basis; of course, a measurement can be described according to an arbitrary basis; the probabilities of the outcomes can be computed by first applying the corresponding change-of-basis, followed by the standard basis measurement. Measurements can actually be described much more generally: e.g. we can describe outcomes of measurements of a strict subset of a quantum system-the mathematical formalism to describe the outcomes uses the density matrix formalism, which we do not describe here.\textcite{Broadbent2016}



\section{Quantum no-cloning}

One of the most fundamental properties of quantum information is that it is not physically possible, in general, to clone a quantum system  (i.e. there is no physical process that takes as input a single quantum system, and outputs two identical copies of its input). A simple proof follows from the linearity of quantum operations.5 At the intuitive level, this principle is present in almost all of quantum cryptography, since it prevents the classical reconstruction of the description of a given qubit system. For instance, given a single copy $\alpha$ and $\beta$, because measuring disturbs the state. At the formal level, however, we generally of a general qubit $\alpha\vert 0\rangle$+ $\beta\vert 1\rangle$, it is not possible to “extract” a full classical description of require more sophisticated tools to prove the security of quantum cryptography protocols

\section{Quantum entanglement and nonlocality}

Acrucial and rather counter-intuitive feature ofquantummechanics is quantum entanglement, a physical phenomenon that occurs when quantum particles behave in such a way that the quantum state of each particle cannot be described individually. A simple example of such an entangled state are two qubits in the state $\frac{(\vert 00\rangle_{AB} +\vert 11\rangle_{AB})}{\sqrt{2}}$. When Alice measures her
qubit (in system A), she obtains a random bit $a \in {0, 1}$ as outcome and her qubit collapses  to the state $\vert a\rangle_A$ she observed. At the same time, Bob’s qubit (in system B) also collapses to $\vert a\rangle_B$ and hence, a subsequent measurement by Bob yields the same outcome $b = a$. It is important to realize that this collapse of state at Bob’s side occurs simultaneously with Alice’s measurement, but it does not allow the players to send information from Alice to Bob.
\section{Essential properties of polarized photons}

Polarized light can be produced by sending an ordinary
light beam through a polarizing apparatus such as a Polaroid filter or calcite crystal; the beam’s polarization axis is determined by the orientation of the polarizing apparatus in which the beam originates. Generating single polarized photons is also possible, in principle by picking them out of a polarized beam, and in practice by a variation of an experiment of Aspect \textcite{Aspect1982} et al.

Although polarization is a continuous variable, the uncertainty principle forbids measurements on any single photon from revealing more than one bit about its polarization. For example, if a light beam with polarization axis $\alpha$ is sent into a filter oriented at angle $\beta$, the individual photons behave dichotomously and probabilistically, sorbed with the complementary probability $\sin 2(\alpha \beta)$. being transmitted with probability $\cos 2(\alpha \beta)$ and ab The photons behave deterministically only when the two axes are parallel (certain transmission) or perpendicular (certain absorption). 

If the two axes are not perpendicular, so that some photons are transmitted, one might hope to learn additional information about $\alpha$ by measuring the transmitted photons again with a polarizer oriented at some third angle; but this is to no avail, because the transmitted photons, in passing through the $\beta$ polarizer, emerge with exactly $\beta$ polarization, having lost all memory of their previous polarization $\alpha$. Another way one might hope to learn more than one bit from a single photon would be not to measure it directly, but rather somehow amplify it into a clone of identically polarized photons, then perform measurements on these; but this hope is also vain, because such cloning can be shown to be inconsistent with the foundations of quantum mechanics\textcite{Wootters1982}.