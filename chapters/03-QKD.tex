\chapter{Quantum Key Distribution}
\label{chap:Quantum Key Distribution}

% QKD is by far the most successful application of quantum information to cryptography. By now, QKD is the main topic of a large number of surveys.
迄今为止,QKD是量子信息在密码学中最成功的应用。到目前为止,QKD是大量调查的主要主题。

\section{Comparsion with Traditional Cryptography}
% In traditional public-key cryptography, trapdoor functions are used to conceal the meaning of messages between two users from a passive eavesdropper, despite the lack of any initial shared secret information between the two users. 
在传统的公共密钥密码学中,尽管两个用户之间没有任何初始共享的秘密信息,但陷阱门功能用于从被动窃听者隐藏两个用户之间的消息含义。

% In quantum public key distribution, the quantum channel is not used directly to send meaningful messages, but is rather used to transmit a supply of random bits between two users who share no secret information initially, in such a way that the users, by subsequent consultation over an ordinary non-quantum channel subject to passive eavesdropping, can tell with high probability whether the original quantum transmission has been disturbed in transit, as it would be by an eavesdropper (it is the quantum channel’s peculiar virtue to compel eavesdropping to be active). If the transmission has not been disturbed, they agree to use these shared secret bits in the well-known way as a one-time pad to conceal the meaning of subsequent meaningful communications, or for other cryptographic applications (e.g. authentication tags) requiring shared secret random information. If transmission has been disturbed, they discard it and try again, deferring any meaningful communications until they have succeeded in transmitting enough random bits through the quantum channel to serve as a one-time pad.
在量子公共密钥分发中,量子通道不是直接用于发送有意义的消息,而是用于在最初不共享秘密信息的两个用户之间传输随机位的提供,以这种方式,用户可以通过随后的协商在一个普通的非量子通道上进行被动窃听,可以很容易地判断原始量子传输是否在传输过程中受到干扰,就像窃听者那样(强迫窃听是主动的,这是量子通道的特质)。如果传输没有受到干扰,则他们同意以众所周知的方式将这些共享机密比特用作一次性密码,以掩盖后续有意义的通信的含义,或者用于需要共享机密的其他密码应用程序(例如,身份验证标签)随机信息。如果传输受到干扰,他们将放弃并重试,推迟任何有意义的通信,直到他们成功通过量子信道传输了足够的随机比特以用作一次性填充。

\section{Protocol}
% one user (‘Alice’) chooses a random bit string and a random sequence of polarization bases (rectilinear or diagonal). She then sends the other user (‘Bob’) a train of photons, each representing one bit of the string in the basis chosen for that bit position, a horizontal or 45-degree photon standing for a binary zero and a vertical or 135-degree photon standing for a binary 1. 

% As Bob receives the photons he decides, randomly for each photon and independently of Alice, whether to measure the photon’s rectilinear polarization or its diagonal polarization, and interprets the result of the measurement as a binary zero or one. As explained in the previous section a random answer is produced and all information lost when one attempts to measure the rectilinear polarization of a diagonal photon, or vice versa. Thus Bob obtains meaningful data from only half the photons he detects—those for which he guessed the correct polarization basis. Bob’s information is further degraded by the fact that, realistically, some of the photons would be lost in transit or would fail to be counted by Bob’s imperfectly-efficient detectors.

% Subsequent steps of the protocol take place over an ordinary public communications channel, assumed to be susceptible to eavesdropping but not to the injection or alteration of messages. Bob and Alice first determine, by public exchange of messages, which photons were successfully received and of these which were received with the correct basis. If the quantum transmission has been undisturbed, Alice and Bob should agree on the bits encoded by these photons, even though this data has never been discussed over the public channel. Each of these photons, in other words, presumably carries one bit of random information (e.g. whether a rectilinear photon was vertical or horizontal) known to Alice and Bob but to no one else.
一个用户(“Alice”)选择一个随机的位串和一个随机的极化基序列(直线或对角线)。然后,她向另一位用户(“Bob”)发送一列光子,每个光子代表为该位位置选择的基础上的字符串的一位,水平或45度光子代表二进制零,垂直或135-度光子代表二进制1。

当Bob接收光子时,他决定独立于每个Alice,对每个光子随机地测量光子的直线极化还是对角极化,并将测量结果解释为二进制零或一。如前一节所述,当尝试测量对角光子的直线偏振时,会产生随机答案,并且所有信息都会丢失,反之亦然。因此,Bob仅从他检测到的一半光子中获得有意义的数据,这些光子他猜出了正确的极化基础。实际上,有些光子在传输中会丢失或无法由Bob的效率不佳的检测器计数,从而使Bob的信息进一步恶化。

该协议的后续步骤在一个普通的公共通信信道上进行,假定该信道易于被窃听,但不易于消息的注入或更改。Bob和Alice首先通过公开交换消息来确定成功接收了哪些光子,以及以正确的基础接收了哪些光子。如果量子传输没有受到干扰,即使从未在公共信道上讨论过该数据,Alice和Bob也应就这些光子编码的比特达成一致。换句话说,这些光子中的每一个可能携带一点Alice和Bob所知的随机信息(例如,直线光子是垂直还是水平)。

% Because of the random mix of rectilinear and diagonal photons in the quantum transmission, any eavesdropping carries the risk of altering the transmission in such a way as to produce disagreement between Bob and Alice on some of the bits on which they think they should agree. Specifically, it can be shown that no measurement on a photon in transit, by an eavesdropper who is informed of the photon’s original basis only after he has performed his measurement, can yield more than $\frac{1}{2}$ expected bits of information about the key bit encoded by that photon; and that any such measurement yielding $b$ bits of expected information ($ b \leq \frac{1}{2}$) must induce a disagreement with probability at least $\frac{b}{2}$ if the measured photon, or an attempted forgery of it, is later re-measured in its original basis. (This optimum tradeoff occurs, for example, when the eavesdropper measures and retransmits all intercepted photons in the rectilinear basis, thereby learning the correct polarizations of half the photons and inducing disagreements in $\frac{1}{4}$ of those that are later re-measured in the original basis.)\textcite{Bennett2014}
由于量子传输中直线形光子和对角线光子的随机混合,任何窃听都存在改变传输率的风险,从而使鲍勃和Alice在他们认为应该同意的某些位上产生分歧。具体来说,可以证明窃听者在传输光子时没有任何测量结果,窃听者只有在完成测量后才获悉光子的原始基础,才能产生超过$\frac{1}{2}$的预期比特关于由该光子编码的密钥位的信息;并且,如果这样的测量产生了$b$位的预期信息($b\leq\frac{1}{2}$),则如果测量的光子必须引起概率至少为$ \frac{b}{2}$的分歧。 ,或尝试对其进行伪造,随后将在其原始基础上重新进行测量。(例如,当窃听者以直线方式测量并重新传输所有被拦截的光子,从而了解一半光子的正确极化并在其中的$\frac{1}{4}$以后在原始基础上重新测量的信息中引起分歧时,就会发生这种最佳折衷。)\textcite{Bennett2014}

% Alice and Bob can therefore test for eavesdropping by
% publicly comparing some of the bits on which they think they should agree, though of course this sacrifices the secrecy of these bits. The bit positions used in this comparison should be a random subset (say one third) of the correctly received bits, so that eavesdropping on more than a few photons is unlikely to escape detection. If all the comparisons agree, Alice and Bob can conclude that the quantum transmission has been free of significant eavesdropping, and those of the remaining bits that were sent and received with the same basis also agree, and can safely be used as a one-time pad for subsequent secure communications over the public channel. When this one-time pad is used up, the protocol is repeated to send a new body of random information over the quantum channel.
因此,Alice和Bob可以通过以下方式测试是否进行了监听
公开比较他们认为应该同意的一些比特,尽管这当然牺牲了这些比特的保密性。 在此比较中使用的位位置应该是正确接收的位的一个随机子集(例如三分之一),这样,窃听多个光子就不可能逃脱检测。 如果所有比较都同意,那么Alice和Bob可以得出结论,量子传输没有明显的窃听,并且在相同基础上发送和接收的其余比特中的那些也同意,并且可以安全地用作one-time pad,以便随后通过公共渠道进行安全通信。当one-time pad耗尽时,将重复该协议以在量子通道上发送新的随机信息主体。

如果Alice和鲍勃事先就一个小秘密密钥达成了共识,则该方案中的公共(非量子)信道免于主动窃听的需求便可以放宽,他们将其用于创建Wegman–Carter身份验证标签\cite{Goldreich1991},以用于其消息传递。公共频道。更详细地讲,Wegman-Carter多消息身份验证方案使用一个小的随机密钥为任意大的消息生成依赖于消息的“标签”(类似于校验和),以使窃听者不知道该密钥能够生成任何其他有效消息-标记对的可能性很小。因此,标签提供了证明消息是合法的证据,并且不是由不知道密钥的人生成或更改的。 (密钥位已在Wegman-Carter方案中逐渐用尽,在不损害系统可证明的安全性的情况下无法重用;但是,在本应用中,这些密钥位可以由成功通过量子信道传输的新鲜随机位代替。)窃听者仍然可以通过抑制公共信道中的消息来阻止通信,当然,他可以通过抑制或过度干扰通过量子信道发送的光子来阻止通信。但是,无论哪种情况,Alice和Bob都将很可能得出结论,即他们的秘密通信被压制了,而事实上却并非如此,他们不会误认为他们的通信是安全的。
% The need for the public (non-quantum) channel in this scheme to be immune to active eavesdropping can be relaxed if Alice and Bob have agreed beforehand on a small secret key, which they use to create Wegman– Carter authentication tags \cite{Goldreich1991} for their messages over the public channel. In more detail the Wegman–Carter multiple-message authentication scheme uses a small random key to produce a message-dependent ‘tag’ (rather like a check sum) for an arbitrarily large message, in such a way that an eavesdropper ignorant of the key has only a small probability of being able to generate any other valid message–tag pairs. The tag thus provides evidence that the message is legitimate, and was not generated or altered by someone ignorant of the key. (Key bits are gradually used up in the Wegman–Carter scheme, and cannot be reused without compromising the system’s provable security; however, in the present application, these key bits can be replaced by fresh random bits successfully transmitted through the quantum channel.) The eavesdropper can still prevent communication by suppressing messages in the public channel, as of course he can by suppressing or excessively perturbing the photons sent through the quantum channel. However, in either case, Alice and Bob will conclude with high probability that their secret communications are being suppressed, and will not be fooled into thinking their communications are secure when in fact they’re not.


% \section{Quantum Coin Tossing} 
% Quantum coin tossing is a scheme involving classi-
% cal and quantum messages which is secure against traditional kinds of cheating, even by an opponent with unlimited computing power. Ironically, it can be subverted by a still subtler quantum phenomenon, the so-called Einstein– Podolsky–Rosen effect. This threat is merely theoretical, because it requires perfect efficiency of storage and detection of photons, which though not impossible in principle is far beyond the capabilities of current technology. The honestly-followed protocol, on the other hand, could be realized with current technology

% \begin{enumerate}
%     \item Alice chooses randomly one basis (say rectilinear) and a sequence of random bits (one thousand should be sufficient). She then encodes her bits as a sequence of photons in this same basis, using the same coding scheme as before. She sends the resulting train of polarized photons to Bob.
%     \item Bob chooses, independently and randomly for each photon, a sequence of reading bases. He reads the photons accordingly, recording the results in two tables, one of rectilinearly received photons and one of diagonally received photons. Because of losses in his detectors and in the transmission channel, some of the photons may not be received at all, resulting in holes in his tables. At this time, Bob makes his guess as to which basis Alice used, and announces it to Alice. He wins if he guessed correctly, loses otherwise.
%     \item Alice reports to Bob whether he won, by telling him which basis she had actually used. She certifies this information by sending Bob, over a classical channel, her entire original bit sequence used in step 1.
%     \item Bob verifies that no cheating has occurred by comparing Alice’s sequence with both his tables. There should be perfect agreement with the table corresponding to Alice’s basis and no correlation with the other table. In our example, Bob can be confident that Alice’s original basis was indeed rectilinear as claimed.
% \end{enumerate}
\section{实用化点对点量子通信}
该方法要求随机改变相干态脉冲强度而测出单光子计数率. 以此为输入参数提炼出最终码. 采用该法所得最终码, 其安全性与用理想单光子源所获最终码等价. 对于弱相干态光源所发射的 脉冲, 有一部分是多光子脉冲, 一部分是单光子脉冲. 诱骗态方法的主要功能是测算在接受端 Bob 的 探测结果 (初始码) 中, 有多少起源于发射端 (Alice 端) 光源的单光子脉冲, 多少起源于发射端的多光 子脉冲. 基于这个至关重要的参数, 就可以提炼出安全的最终码, 其安全性等同于只采用了由发射端 单光子脉冲产生的那部分初始码而抛弃了多光子脉冲产生的那部分初始码. 在安全性方面最后的效果 就等同于使用了理想单光子源.

2003 年, 美国西北大学黄元瑛博士提出了在量子密码理论实用化上具有革命性的 Decoy-State 思想 \cite{Hwang2003} 用以解决光子数分离攻击. 可是黄的结果尚不能立即实用于现有真实系统, 清华大学王向斌教授于 2005 年的理论研究 \cite{Wang2005} 表明, 采用三强度随机切换的诱骗信号量子密码方案可以准确侦察出任 何窃听行为, 包括所谓的光子数分离攻击, 并可立即实用于现有真实系统, 其中包含通道噪声, 大损耗 等. 三强度诱骗信号方法可以让合法用户计算出至关重要的参量: 多光子脉冲份额的上限值. 有此上 限值, 结合前人理论结果, 便可以获得绝对安全的最终码. 由该理论给出的关键计算公式, 诱骗态方法 具有了立即的实用价值 (见图 4). 这也使得量子密钥分发有可能成为整个量子信息领域最先走入社会 实用的分支.

诱骗态方法的首个实验由清华大学和中国科技大学等单位的联合团队完成 \cite{Peng2007}, 这也成为历史上首次超过 100 km 的安全量子密钥分发. 同一时期的实验还有美国橡树岭国家实验室与美国国家标准 局团队的合作实验、维也纳大学等单位的实验等. 此后, 中国科技大学结合光开关技术, 把诱 骗态方法用于量子网络, 先后实现了 3 节点与 5 节点的量子网络安全通信 \cite{Chen2009}. 迄今为止, 基于诱骗态 方法的量子密钥分发已经至少获得世界主要研究机构近 20 个公开发表的在不同条件下的实验证实. 尽管诱骗态方法未必就是唯一方法, 由于其安全性和实用性, 事实上, 诱骗态方法已经成为当前 量子密码走向实际应用的最重要方法. 近年来, 中国科学家们致力于参与这一主战场的研究, 在实验 与理论方面取得国际领先的广泛成果. 自清华 — 中科大联合团队 2007 年在国际上率先利用诱骗态 手段实现了绝对安全距离超过一百公里的量子密钥分发 \cite{Peng2007} 以来, 中国科技大学潘建伟小组又于 2010 年率先实现绝对安全距离达 200 km 的量子密钥分发, 为目前国际上绝对安全量子密钥分发最远距 离. 他们还采用光开关技术, 于 2008 年 10 月初完成了诱骗态量子密钥分发的 “光量子电话网”\cite{Chen2009}(此 前国内外其他小组的量子密码网络的实验因为没有采用诱骗态方法而不安全). 清华大学王向斌小组 则通过系统化的理论研究已经证明即便光源强度有较大涨落诱骗态方法依然有效, 给出了相关安全成 码的计算公式

\section{量子网络通信}
辅以光开关技术后, 诱骗态方法还可用以实现量子通信网络. 由于没有量子存储器, 这种网络的 量子密钥分发距离不能超越点对点的量子密钥分发距离. 然而, 网络上的任何两个用户可以通过光开关切换实现量子密钥分发. 我国在 2009 年实现了 3 节点的链状量子通信网络 \cite{Chen2009}, 为世界上首个基于 诱骗态方案的量子语音通信网络系统, 实现了实时网络通话和三方对讲功能, 演示了无条件安全的量 子通信的可实用化. 此成果很快被美国《Science》杂志以 “量子电话” 为题进行了报道, 亦被欧洲物理 学会《物理世界》以 “中国诞生量子网络” 为题做了专题报道. 随后, 又实现了 5 节点城域量子通信 网络\cite{Chen2010}, 是国际上首个全通型的量子通信网络, 各节点全部演示了安全的语音通信. 值得指出的是, 与 欧洲 SECOQC 网络以及 Tokyo QKD network 不同, 这两个量子通信网络是基于诱骗态方案的成熟技 术, 追求并逐步实现满足信息论定义下严格安全性要求的实用性, 而不是欧洲、美国和日本同行所做 的多种技术的混合展示. 我国此类小规模的演示性网络还有多节点的城域量子政务网.