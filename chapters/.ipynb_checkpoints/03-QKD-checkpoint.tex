\chapter{Quantum Key Distribution}
\label{chap:Quantum Key Distribution}

QKD is by far the most successful application of quantum information to cryptography. By now, QKD is the main topic of a large number of surveys.


\section{Comparsion with Traditional Cryptography}
In traditional public-key cryptography, trapdoor functions are used to conceal the meaning of messages between two users from a passive eavesdropper, despite the lack of any initial shared secret information between the two users. 

In quantum public key distribution, the quantum channel is not used directly to send meaningful messages, but is rather used to transmit a supply of random bits between two users who share no secret information initially, in such a way that the users, by subsequent consultation over an ordinary non-quantum channel subject to passive eavesdropping, can tell with high probability whether the original quantum transmission has been disturbed in transit, as it would be by an eavesdropper (it is the quantum channel’s peculiar virtue to compel eavesdropping to be active). If the transmission has not been disturbed, they agree to use these shared secret bits in the well-known way as a one-time pad to conceal the meaning of subsequent meaningful communications, or for other cryptographic applications (e.g. authentication tags) requiring shared secret random information. If transmission has been disturbed, they discard it and try again, deferring any meaningful communications until they have succeeded in transmitting enough random bits through the quantum channel to serve as a one-time pad.

\section{Protocol}
one user (‘Alice’) chooses a random bit string and a random sequence of polarization bases (rectilinear or diagonal). She then sends the other user (‘Bob’) a train of photons, each representing one bit of the string in the basis chosen for that bit position, a horizontal or 45-degree photon standing for a binary zero and a vertical or 135-degree photon standing for a binary 1. 

As Bob receives the photons he decides, randomly for each photon and independently of Alice, whether to measure the photon’s rectilinear polarization or its diagonal polarization, and interprets the result of the measurement as a binary zero or one. As explained in the previous section a random answer is produced and all information lost when one attempts to measure the rectilinear polarization of a diagonal photon, or vice versa. Thus Bob obtains meaningful data from only half the photons he detects—those for which he guessed the correct polarization basis. Bob’s information is further degraded by the fact that, realistically, some of the photons would be lost in transit or would fail to be counted by Bob’s imperfectly-efficient detectors.

Subsequent steps of the protocol take place over an ordinary public communications channel, assumed to be susceptible to eavesdropping but not to the injection or alteration of messages. Bob and Alice first determine, by public exchange of messages, which photons were successfully received and of these which were received with the correct basis. If the quantum transmission has been undisturbed, Alice and Bob should agree on the bits encoded by these photons, even though this data has never been discussed over the public channel. Each of these photons, in other words, presumably carries one bit of random information (e.g. whether a rectilinear photon was vertical or horizontal) known to Alice and Bob but to no one else.

Because of the random mix of rectilinear and diagonal photons in the quantum transmission, any eavesdropping carries the risk of altering the transmission in such a way as to produce disagreement between Bob and Alice on some of the bits on which they think they should agree. Specifically, it can be shown that no measurement on a photon in transit, by an eavesdropper who is informed of the photon’s original basis only after he has performed his measurement, can yield more than $\frac{1}{2}$ expected bits of information about the key bit encoded by that photon; and that any such measurement yielding $b$ bits of expected information ($ b \leq \frac{1}{2}$) must induce a disagreement with probability at least $\frac{b}{2}$ if the measured photon, or an attempted forgery of it, is later re-measured in its original basis. (This optimum tradeoff occurs, for example, when the eavesdropper measures and retransmits all intercepted photons in the rectilinear basis, thereby learning the correct polarizations of half the photons and inducing disagreements in $\frac{1}{4}$ of those that are later re-measured in the original basis.)\textcite{Bennett2014}

Alice and Bob can therefore test for eavesdropping by
publicly comparing some of the bits on which they think they should agree, though of course this sacrifices the secrecy of these bits. The bit positions used in this comparison should be a random subset (say one third) of the correctly received bits, so that eavesdropping on more than a few photons is unlikely to escape detection. If all the comparisons agree, Alice and Bob can conclude that the quantum transmission has been free of significant eavesdropping, and those of the remaining bits that were sent and received with the same basis also agree, and can safely be used as a one-time pad for subsequent secure communications over the public channel. When this one-time pad is used up, the protocol is repeated to send a new body of random information over the quantum channel.

The need for the public (non-quantum) channel in this scheme to be immune to active eavesdropping can be relaxed if Alice and Bob have agreed beforehand on a small secret key, which they use to create Wegman– Carter authentication tags \cite{Goldreich1991} for their messages over the public channel. In more detail the Wegman–Carter multiple-message authentication scheme uses a small random key to produce a message-dependent ‘tag’ (rather like a check sum) for an arbitrarily large message, in such a way that an eavesdropper ignorant of the key has only a small probability of being able to generate any other valid message–tag pairs. The tag thus provides evidence that the message is legitimate, and was not generated or altered by someone ignorant of the key. (Key bits are gradually used up in the Wegman–Carter scheme, and cannot be reused without compromising the system’s provable security; however, in the present application, these key bits can be replaced by fresh random bits successfully transmitted through the quantum channel.) The eavesdropper can still prevent communication by suppressing messages in the public channel, as of course he can by suppressing or excessively perturbing the photons sent through the quantum channel. However, in either case, Alice and Bob will conclude with high probability that their secret communications are being suppressed, and will not be fooled into thinking their communications are secure when in fact they’re not.


\section{Quantum Coin Tossing} 
Quantum coin tossing is a scheme involving classi-
cal and quantum messages which is secure against traditional kinds of cheating, even by an opponent with unlimited computing power. Ironically, it can be subverted by a still subtler quantum phenomenon, the so-called Einstein– Podolsky–Rosen effect. This threat is merely theoretical, because it requires perfect efficiency of storage and detection of photons, which though not impossible in principle is far beyond the capabilities of current technology. The honestly-followed protocol, on the other hand, could be realized with current technology

\begin{enumerate}
    \item Alice chooses randomly one basis (say rectilinear) and a sequence of random bits (one thousand should be sufficient). She then encodes her bits as a sequence of photons in this same basis, using the same coding scheme as before. She sends the resulting train of polarized photons to Bob.
    \item Bob chooses, independently and randomly for each photon, a sequence of reading bases. He reads the photons accordingly, recording the results in two tables, one of rectilinearly received photons and one of diagonally received photons. Because of losses in his detectors and in the transmission channel, some of the photons may not be received at all, resulting in holes in his tables. At this time, Bob makes his guess as to which basis Alice used, and announces it to Alice. He wins if he guessed correctly, loses otherwise.
    \item Alice reports to Bob whether he won, by telling him which basis she had actually used. She certifies this information by sending Bob, over a classical channel, her entire original bit sequence used in step 1.
    \item Bob verifies that no cheating has occurred by comparing Alice’s sequence with both his tables. There should be perfect agreement with the table corresponding to Alice’s basis and no correlation with the other table. In our example, Bob can be confident that Alice’s original basis was indeed rectilinear as claimed.
\end{enumerate}