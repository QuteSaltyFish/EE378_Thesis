\chapter{Introduction}
\label{chap:introduction}

\section{Overview}
“最近的 16 公里量子态隐形传输的成功试验表明, 中国将有能力建立起卫星与地面的安全量子通
信网络。”—— 美国《时代周刊》在 “爆炸性新闻” 栏目中以 “中国量子科学的飞跃” 为题, 对 2010 年 中国科技大学与清华大学合作完成的 16 公里量子态隐形传输试验进行了评论。 相比于经典通信, 量子 通信究竟有哪些优势, 有哪些应用, 源于何种原理以及方法和技术手段等, 无疑是大家所关心的。 我们 将在此介绍量子通信的基本概念与方法、技术现状, 以及未来应用前景。 量子通信的基本思想主要由 Bennett 等于 20 世纪 80 年代和 90 年代起相继提出, 主要包括量子
密钥分发 (quantum key distribution, QKD)\cite{Bennett2014} 和量子态隐形传输 (quantum teleportation)\cite{Bennett1993}。 量子密钥分发可以建立安全的通信密码, 通过一次一密的加密方式可以实现点对点方式的安全经典通信。 这里的安全性是在数学上已经获得严格证明的安全性, 这是经典通信迄今为止做不到的。 现有的量子密钥分发技术可以实现百公里量级的量子密钥分发, 辅以光开关等技术, 还可以实现量子密钥分发网络。 量子态隐形传输是基于量子纠缠态的分发与量子联合测量, 实现量子态 (量子信息) 的空间转移而又不移动量子态的物理载体, 这如同将密封信件内容从一个信封内转移到另一个信封内而又不移 动任何信息载体自身。 这在经典通信中是无法想象的事。 基于量子态隐形传输技术和量子存储技术的量子中继器可以实现任意远距离的量子密钥分发及网络。
% Rapid development of supercomputers and the prospect of quantum computers are posing increasingly serious threats to the security of communication。 Using the principles of quantum cryptography, quantum communication offers provable security of communication and is a promising solution to counter such threats。 Quantum cryptography that does not depend on the computer capacity of the adversary but is absolutely guaranteed by the laws of quantum physics, although it is in an initial stage, it is necessary to motivate the work for research purposes in the academic world, industry and society in general is taken as a solid alternative of security。

% There is no doubt that QKD has taken the spotlight in terms of the use of quantum information for cryptography (in fact, so much that the term “quantum cryptography” is often equated with QKD—a misconception that we aim to rectify here!); yet there exist many other uses ofquantum information in cryptography。 What ismore, quantum information opens up the cryptographic landscape to allowfunctionalities that do not exist using classical1 information alone, for example uncloneable quantum money。 We note, however, that the use of quantum information in cryptography has its limitations and challenges。 For instance, we know that quantum information alone is insufficient to implement information-theoretically secure bit commitment; and that a proof technique called rewinding (which is commonly used in establishing a zero-knowledge property for a protocol) does not directly carry over to the quantum world and must re-visited in light of quantum information。

超级计算机的快速发展和量子计算机的前景对通信的安全性构成越来越严重的威胁。利用量子密码学的原理,量子通信提供了可证明的通信安全性,并且是应对此类威胁的有希望的解决方案。量子密码学不依赖对手的计算机能力,但由量子物理学定律绝对地保证,尽管它处于起步阶段,但有必要在学术界,工业界和社会中激励研究工作通常,将其作为安全性的可靠替代方案。

毫无疑问,QKD在使用量子信息进行密码学方面引起了人们的关注(事实上,“量子密码学”一词经常与QKD等同,这是一种误解), 然而,量子信息在密码学中还有许多其他用途。此外,量子信息打开了密码领域,允许仅使用经典信息就无法实现的功能,例如不可克隆的量子货币。但是,我们注意到,密码学中使用量子信息有其局限性和挑战。例如,我们知道仅量子信息不足以实现信息理论上安全的比特承诺;并且一种称为倒带的证明技术(通常用于建立协议的零知识特性)不会直接延续到量子世界,而必须根据量子信息重新进行研究。

% \section{Citation styles}

% These are the different citation styles for author-year。

% The standard \verb=\cite= command produces the following output: \cite{clarke1990rendezvous}。

% The \verb=\textcite= command produces the following output: \textcite{clarke1990rendezvous}。


% The \verb=\parencite= command produces the following output: \parencite{clarke1990rendezvous}。

% The \verb=\footcite= command produces a footnote citation \footcite{clarke1990rendezvous}。

% \clearpage          % To start a new page

% \section{Figures}

% Always prefer to float figures to the top of the page using the [t] option in the figure environment。

% \begin{figure}[t]
%     \centering
%     \includegraphics{figures/blackbox。jpeg}
%     \caption{This is a black box}
%     \label{fig:fig_1}
% \end{figure}

% \clearpage

% \section{Tables}

% Use vertical lines sparingly in tables。 They're unnecessary bloat。 Write the code for tables in a separate tex file, and include it in when required。 Also preferrably use sans serif font for tables (because of their information density) using \texttt{\\sffamily} in table definition。

% \begin{table}[!ht]
% \small
% \centering
% \sffamily
% \input{tables/table。tex}
% \caption{This is a table}
% \label{tab:table}
% \vspace{-5mm}
% \end{table}

