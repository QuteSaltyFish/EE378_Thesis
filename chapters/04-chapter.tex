\chapter{限制与挑战}
\label{chap:限制与挑战}

\section{Impossibility of quantum bit commitment}
在第一个QKD协议\cite{Bennett2014}发布后的10年中,只有少数密码学家从事量子密码学工作。下面列出了主要的不可能论据。

首先,考虑以下不可能完全安全地实现经典比特承诺的草图:假设存在这样的协议。然后,根据信息理论安全性要求,在承诺阶段结束时,鲍勃对协议的看法必须独立于b(否则,协议但是相位,两者都被Bob接受。因此,位承诺不能绑定。这种独立性意味着Alice可以选择在显示中显示b = 0或b =1。有趣的是,相同的证明该结构适用于量子情况,尽管通过调用一些更多的技术工具也是如此。即,我们首先考虑该协议的一个纯净版本,该版本包含所有在量子级别上起作用的各方(通过标准,一个统一的过程代替了测量)接下来,通过信息论的隐藏特性,必须确定鲍勃在提交阶段结束时持有的还原量子态。 ical,无论b = 0还是b =1。此条件足以破坏她的系统,以便重新创建与b = 0或b = 1一致的关节状态。13因此,她可以选择稍后再打开b = 0或b = 1时,Bob会接受:承诺方案不能具有约束力。

回到有关量子比特承诺的原始论文,我们注意到绑定属性的定义中的一个细微之处是安全性虚假主张的起源:尽管确实如此,该协议使得Alice无法同时保存消息。 (b = 0和b = 1),这不足以证明安全性,因为实际上Alice能够延迟会宣布对b = 0且b = 1的承诺(因此可以选择开放她选择承诺,直到协议结束–此时她可以选择以b = 0或b = 1打开。
% The 10-year period following the publication ofthe first QKD protocol \cite{Bennett2014} saw only a handful of cryptographers working in quantum cryptography。 The main impossibility argument are listed below。

% First, consider the following sketch of impossibility for perfectly secure classical bit commitment: suppose such a protocol exists。 Then by the informationtheoretic security requirement, at the end ofthe commitment phase,Bob’s viewofthe protocol must be independent of b (since, otherwise, the protocol would not be perfectly hiding)。 But phase, with both being accepted by Bob。 Hence, the bit commitment cannot be binding。 this independence implies that Alice can choose to reveal either b = 0or b = 1inthe reveal It is interesting that the same proof structure is applicable to the quantum case, albeit by invoking some slightly more technical tools。 Namely, we first consider a purified version of the protocol, which consists in all parties acting at the quantum level (measurements are replaced by a unitary process via a standard technique)。 Next, by the information-theoretic hiding property, the reduced quantum state that Bob holds at the end of the commit phase must be identical, whether b = 0or b = 1。 This condition is enough to break the binding her system in order to re-create a joint state consistent with either b = 0, or b = 1, at her choosing。13 Hence, she can chose to open either b = 0or b = 1 at a later time, and Bob will accept: the commitment scheme cannot be binding。

% Going back to the original paper on quantum bit commitment, we note that a subtlety in the definition of the binding property is the origin of the false claim of security: while it is true that the protocol is such that Alice is unable to simultaneously hold messages that b = 0 and b = 1), this is insufficient to prove security, since in fact Alice is able to delay would unveil a commitment to b = 0 and as b = 1 (and thus, to be able to choose to open her choice of commitment until the very end of the protocol—at which point she can choose to open as either b = 0 or b = 1。
回到有关量子比特承诺的原始论文,我们注意到绑定属性的定义中的一个细微之处是安全性虚假主张的起源:尽管确实如此,该协议使得Alice无法同时保存消息。 (b = 0和b = 1),这不足以证明安全性,因为实际上Alice能够延迟会宣布对b = 0且b = 1的承诺(因此可以选择开放她 选择承诺,直到协议结束–此时她可以选择以b = 0或b = 1打开。


\section{Impossibility of secure two-party computation using quantum communication}
鉴于不可能实现量子比特承诺,下一个要问的问题是:是否存在可以使用量子通信安全地实现的经典通信?实际上,OT的可能性被认为是一个未解决的问题\cite{Montrdal1996}。不幸的是,这种希望很快就破灭了,因为Lo\cite{Lo2008}在单边计算中给出了不可能的结果(其中只有一方收到输出)。 1出2的OT的不可能性—证明技术紧跟为量子比特承诺的不可能性开发的技术。科尔贝花了近10年的时间才展示了两个OT的第一个不可能结果。侧面的计算,即爱丽丝总是可以获得比鲍勃的输入更多的信息,而不是该函数的值所暗示的信息。类似地,萨尔维尔,沙夫纳和索塔科娃证明,任何非平凡的原始信息的量子协议都会泄漏信息给更糟糕的是,即使在受信方的帮助下,量子通信协议也无法“放大”任何原语的加密能力。

Buhrman,Christandl和Schaffner\cite{Salvail2015}加强了上述不可能的结果,通过证明任何量子协议中的泄漏本质上都可以像人们想象的那样严重:即使在近似正确性和安全性的情况下,如果一个协议对鲍勃“安全”,那么它对爱丽丝也是完全不安全的(在某种意义上,她可以为所有可能的输入计算计算的输出)。由于不可能,导致通用可组合性(UC)框架。
% Given the impossibility of quantum bit commitment, the next question to ask is: are there any classical primitives that may be implemented securely using quantum communication? In fact, the possibility for OT was stated as a open problem in \cite{Montrdal1996}。 Unfortunately, this hope was shattered rather quickly, as impossibility results were given by Lo in \cite{Lo2008} for one- sided computations (where only one party receives output)。 This result already shows the impossibility of 1-out-of-2 OT—the proof technique follows closely the technique developed for the impossibility of quantum bit commitment (see Sect。 4。1)。 It took almost 10years until Colbeck showed the first impossibility result for two-sided computations, namely that Alice can always obtain more information about Bob’s input than what is implied by the value of the function。 In a similar vein, Salvail, Schaffner and Sotakova proved in \cite{Salvail2015} that any quantum protocol for a non-trivial primitive leaks information to a dishonest player。 What is worse, even with the help of a trusted party, the cryptographic power of any primitive cannot be “amplified” by a quantum-communication protocol。
% Buhrman, Christandl and Schaffner have strengthened the above impossibility results
% by showing that the leakage in any quantum protocol is essentially as bad as one can imagine: even in the case of approximate correctness and security, if a protocol is “secure” against Bob, then it is completely insecure against Alice (in the sense that she can compute the output of the computation for all of her possible inputs)。 For impossibility results in the universal composability (UC) framework。

\section{Zero-knowledge against quantum adversaries: “quantum rewinding”}
% Zero-knowledge interactive proofs, as introduced by Goldwasser, Micali, and Rackoff  are interactive proofs with the property that the verifier learns nothing from her interaction with the honest prover, beyond the validity ofthe statement being proved。 These proofsystems play an important role in the foundations of cryptography, and are also fundamental building blocks to achieve cryptographic functionalities (see \cite{Goldreich1991} for a survey)。 In zero-knowledge interactive proofs, the notion that the verifier “learns nothing” is for-
% malized via the simulation paradigm: if, for every cheating verifier (interacting in the protocol on a positive instance), there exists a simulator (who does not interact with the prover) such that the output of the verifier is indistinguishable from the output of the simulator, thenwe say that the zero-knowledge property holds。 In the classical world, a common proof technique used for establishing the zero-knowledge property is rewinding: a simulator is typically built by executing the given verifier—except that some computation paths are culled if the ran- dom choices of the verifier are not consistent with the desired effect。 This selection is done by keeping a trace of the interaction, thus, if the interaction is deemed to have followed an incorrect path, the simulation can simply reset the computation (“rewind”) to an earlier part of the computation (see \cite{Goldreich1991} and references therein)。
Goldwasser,Micali和Rackoff引入的零知识交互式证明是交互式证明,其性质是验证者从与诚实证明者的交互中不会学到任何东西,超出了被证明陈述的有效性。这些证明系统在验证中起着重要作用。密码学的基础,并且也是实现密码功能的基本构建块(请参阅\cite{Goldreich1991}调查)。在零知识的交互证明中,验证者“不了解任何内容”的概念是-通过模拟范式进行恶意化:如果对于每个作弊验证者(在肯定实例中与协议进行交互),都存在一个模拟器(不与证明者进行交互),以使验证者的输出与验证者的输出无法区分模拟器,那么我们说零知识属性成立。在古典世界中,用于建立零知识属性的通用证明技术是倒带:通常通过执行给定的验证程序来构建模拟器-除了某些计算路径被剔除外如果验证者的随机选择与预期效果不一致。通过跟踪交互来完成选择,因此,如果认为交互遵循了错误的路径,则仿真可以简单地重置计算(“快退”)到计算的较早部分(请参阅\cite{Goldreich1991}其中的参考)。

% In the quantum setting, such a rewinding approach is impossible: the no-cloning theorem
% tells us that it is not possible, in general, to keep a secondary copy of the transcript in order to return to it later on。 This problem is further aggravated by the fact that, in the most general case, the verifier starts with some auxiliary quantum information (which we do not, in general, know how to re-create)—thus even a “patch” that would emulate the rewinding approach in the simple case would appear to fail in the case of auxiliary quantum information。 We emphasize that the above concerns about the zero-knowledge property are applicable to purely classical protocols: honest parties are completely classical, but we wish to establish the zero-knowledge property against a verifier that may receive, store and process quantum information。
在量子环境中,这种倒带方法是不可能的:无克隆定理告诉我们,一般来说,不可能保留成绩单的第二份副本以便以后再返回。在最一般的情况下,验证者以某种辅助方式开头,这一事实使这一问题进一步恶化。 量子信息(通常我们不知道如何重新创建)-因此,即使是简单的情况下模仿倒带方法的“补丁”,在辅助量子信息的情况下也似乎会失败。 以上关于零知识特性的关注点适用于纯古典协议:诚实的当事人是完全古典的,但是我们希望针对可能接收,存储和处理量子信息的验证者建立零知识特性。

\section{光子数分离攻击}
单光子的不可分割性是量子密码安全性的重要物理基础。 然而, 多光子脉冲不再拥有不可分割性。 例如, 一个包含两个光子的脉冲, 原则上可以被分割为两个单光子脉冲, 所以其安全性基 础就不复存在了, 就会遭受光子数分离攻击, 下面我们来具体介绍下光子数分离攻击 \cite{Huttner1995},\cite{Brassard2000}。 由于量子 通信通道损耗率极大, 对于 100 km 以上的距离, 加上探测效率, 整体效率将小于千分之一。根据理论 证明, 理想单光子源即便在高损耗通道下也是绝对安全的, 可是实际系统使用的弱光在高损耗通道下 则结果完全不同: 窃听者可以冒充通道损耗通过光子数分离攻击而获得全部密码。 如图 3 所示, 为表 述方便, 我们以偏振空间为例。 弱相干态脉冲实际上是单光子与多光子脉冲的概率混合。 即,在所发出 的非真空脉冲中, 有些是单光子的, 有些是多光子的 (2光子,3光子……)。 多光子脉冲即包含了多个 全同偏振光子。 窃听者可将其分离, 自己留下一个, 将剩余光子送到远程合法用户。 对于这些多光子 脉冲, 窃听者可以拥有与合法用户完全一样的偏振光子而不对远程合法用户的光子偏振态造成任何扰 动。 即, 对于多光子脉冲, 窃听者可以拥有 100\%的信息而不被察觉。 窃听者可以选择将所有单光子脉 冲完全吸收而使得远程合法用户的所有比特皆由光源的多光子脉冲产生。 窃听者的行为不会被合法用 户察觉, 因为窃听者可以对每个单独脉冲随时调整通道衰减系数, 从而使得远程合法用户的探测器计 数率等同于高损耗自然通道。

对于 2005 年以前的弱相干态密钥分发实验 \cite{Kimura2004}, 窃听者可获取全部信息而不留下任何痕迹。 事实上, 量子密码发明者之一, Brassard 等 \cite{Huttner1995},\cite{Brassard2000} 早在 2000 年就对弱相干态量子密码实验做出批 评, Brassard 等在其著名论文的摘要部分指出: “Existing experimental schemes (based on weak pulses) currently do not offer unconditional security for the reported distances and signal strength”, 即: “现有基 于 (相干态) 弱脉冲的做法, 据其所报告的距离及所采用的脉冲强度, 并不提供绝对安全性。” Brassard 的这一评论适用于 2005 年以前所有基于弱相干光的量子密钥分发实验 \cite{Kimura2004}。 幸运的是, 于 2005 年起发展起来的诱骗态量子密码理论, 提供了一个基于弱相干光源的安全量子密钥分发方案。

\section{侧信道攻击和木马攻击}
尽管量子通信技术在理论上具有 “无条件安全性”, 但理论方案安全性和实际系统安全性这两个层面之间仍存在一条狭窄但分明的缝隙. 利用量子保密通信系统器件的性能缺陷进行窃听, 或者针对 器件的弱点进行主动攻击都可能削弱甚至破坏量子保密通信系统的安全性. 自 2000 年以来, 随着量 子通信技术的逐步实用化, 实用系统中的安全攻防问题变得越来越重要, 并引起研究者的高度重视. 针 对早期方案和实验技术中的安全性漏洞, 已提出了大量的攻击方案, 如伪装态攻击、相位重映射攻击、 定时侧信道攻击、大脉冲攻击、光学部件高能破坏攻击等. 这些攻击方案, 统称为侧信道攻击和木马 攻击.

“木马攻击” 中的木马是指实际的量子保密通信系统其信号源、接收器以及其他部件有可能存在的某种弱点, 针对这种弱点, 可以设计攻击方案, 主动诱使系统内部信息泄露. 如果不弥补器件的弱点, 这种攻击常常能有效地击破量子保密通信系统的安全性. 比如说 “大脉冲攻击” 法, 由于光学器件总 会有一定反射能力. 窃听者因此向光路中发射高亮度激光. 对于某些量子保密通信系统的实现方案, 被反射回来的光会被系统中的极化或相位调制器调制, 这样, 攻击者就得到了发射方信号态的极化或 相位信息, 而不会引入额外的干扰, 也就不会被发现. 再如 “高能破坏攻击” 使用高亮度激光击毁衰减 器, 破坏了弱相干光源, 随后就可以使用 “分束器攻击” 或者 “分离光子数攻击” 窃取密钥. 主动攻击 法还有 “伪装态攻击”、“相位再映射攻击” 等. 而侧信道攻击法是指量子通信系统可能存在泄漏密钥 信息的侧信道. 侧信道攻击最出名的就是分离光子数攻击, 此外, 最近提出的针对有记忆的装置无关 QKD 系统的攻击就利用了经典协商信道的侧信道泄漏.