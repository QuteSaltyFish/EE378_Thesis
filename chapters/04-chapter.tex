\chapter{limitations and challenges}
\label{chap:limitations and challenges}

\section{Impossibility of quantum bit commitment}

The 10-year period following the publication ofthe first QKD protocol \cite{Bennett2014} saw only a handful of cryptographers working in quantum cryptography. The main impossibility argument are listed below.

First, consider the following sketch of impossibility for perfectly secure classical bit commitment: suppose such a protocol exists. Then by the informationtheoretic security requirement, at the end ofthe commitment phase,Bob’s viewofthe protocol must be independent of b (since, otherwise, the protocol would not be perfectly hiding). But phase, with both being accepted by Bob. Hence, the bit commitment cannot be binding. this independence implies that Alice can choose to reveal either b = 0or b = 1inthe reveal It is interesting that the same proof structure is applicable to the quantum case, albeit by invoking some slightly more technical tools. Namely, we first consider a purified version of the protocol, which consists in all parties acting at the quantum level (measurements are replaced by a unitary process via a standard technique). Next, by the information-theoretic hiding property, the reduced quantum state that Bob holds at the end of the commit phase must be identical, whether b = 0or b = 1. This condition is enough to break the binding her system in order to re-create a joint state consistent with either b = 0, or b = 1, at her choosing.13 Hence, she can chose to open either b = 0or b = 1 at a later time, and Bob will accept: the commitment scheme cannot be binding.

Going back to the original paper on quantum bit commitment, we note that a subtlety in the definition of the binding property is the origin of the false claim of security: while it is true that the protocol is such that Alice is unable to simultaneously hold messages that b = 0 and b = 1), this is insufficient to prove security, since in fact Alice is able to delay would unveil a commitment to b = 0 and as b = 1 (and thus, to be able to choose to open her choice of commitment until the very end of the protocol—at which point she can choose to open as either b = 0 or b = 1.




\section{Impossibility of secure two-party computation using quantum communication}

Given the impossibility of quantum bit commitment, the next question to ask is: are there any classical primitives that may be implemented securely using quantum communication? In fact, the possibility for OT was stated as a open problem in \cite{Montrdal1996}. Unfortunately, this hope was shattered rather quickly, as impossibility results were given by Lo in \cite{Lo2008} for one- sided computations (where only one party receives output). This result already shows the impossibility of 1-out-of-2 OT—the proof technique follows closely the technique developed for the impossibility of quantum bit commitment (see Sect. 4.1). It took almost 10years until Colbeck showed the first impossibility result for two-sided computations, namely that Alice can always obtain more information about Bob’s input than what is implied by the value of the function. In a similar vein, Salvail, Schaffner and Sotakova proved in \cite{Salvail2015} that any quantum protocol for a non-trivial primitive leaks information to a dishonest player. What is worse, even with the help of a trusted party, the cryptographic power of any primitive cannot be “amplified” by a quantum-communication protocol.
Buhrman, Christandl and Schaffner have strengthened the above impossibility results
by showing that the leakage in any quantum protocol is essentially as bad as one can imagine: even in the case of approximate correctness and security, if a protocol is “secure” against Bob, then it is completely insecure against Alice (in the sense that she can compute the output of the computation for all of her possible inputs). For impossibility results in the universal composability (UC) framework.

\section{Zero-knowledge against quantum adversaries: “quantum rewinding”}
Zero-knowledge interactive proofs, as introduced by Goldwasser, Micali, and Rackoff  are interactive proofs with the property that the verifier learns nothing from her interaction with the honest prover, beyond the validity ofthe statement being proved. These proofsystems play an important role in the foundations of cryptography, and are also fundamental building blocks to achieve cryptographic functionalities (see \cite{Goldreich1991} for a survey). In zero-knowledge interactive proofs, the notion that the verifier “learns nothing” is for-
malized via the simulation paradigm: if, for every cheating verifier (interacting in the protocol on a positive instance), there exists a simulator (who does not interact with the prover) such that the output of the verifier is indistinguishable from the output of the simulator, thenwe say that the zero-knowledge property holds. In the classical world, a common proof technique used for establishing the zero-knowledge property is rewinding: a simulator is typically built by executing the given verifier—except that some computation paths are culled if the ran- dom choices of the verifier are not consistent with the desired effect. This selection is done by keeping a trace of the interaction, thus, if the interaction is deemed to have followed an incorrect path, the simulation can simply reset the computation (“rewind”) to an earlier part of the computation (see \cite{Goldreich1991} and references therein).

In the quantum setting, such a rewinding approach is impossible: the no-cloning theorem
tells us that it is not possible, in general, to keep a secondary copy of the transcript in order to return to it later on. This problem is further aggravated by the fact that, in the most general case, the verifier starts with some auxiliary quantum information (which we do not, in general, know how to re-create)—thus even a “patch” that would emulate the rewinding approach in the simple case would appear to fail in the case of auxiliary quantum information. We emphasize that the above concerns about the zero-knowledge property are applicable to purely classical protocols: honest parties are completely classical, but we wish to establish the zero-knowledge property against a verifier that may receive, store and process quantum information.