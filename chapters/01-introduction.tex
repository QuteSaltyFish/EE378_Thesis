\chapter{Introduction}
\label{chap:introduction}
\section{Introduction}
Rapid development of supercomputers and the prospect of quantum computers are posing increasingly serious threats to the security of communication. Using the principles of quantum cryptography, quantum communication offers provable security of communication and is a promising solution to counter such threats. Quantum cryptography that does not depend on the computer capacity of the adversary but is absolutely guaranteed by the laws of quantum physics, although it is in an initial stage, it is necessary to motivate the work for research purposes in the academic world, industry and society in general is taken as a solid alternative of security
% \section{Citation styles}

% These are the different citation styles for author-year.

% The standard \verb=\cite= command produces the following output: \cite{clarke1990rendezvous}.

% The \verb=\textcite= command produces the following output: \textcite{clarke1990rendezvous}.


% The \verb=\parencite= command produces the following output: \parencite{clarke1990rendezvous}.

% The \verb=\footcite= command produces a footnote citation \footcite{clarke1990rendezvous}.

% \clearpage          % To start a new page

% \section{Figures}

% Always prefer to float figures to the top of the page using the [t] option in the figure environment.

% \begin{figure}[t]
%     \centering
%     \includegraphics{figures/blackbox.jpeg}
%     \caption{This is a black box}
%     \label{fig:fig_1}
% \end{figure}

% \clearpage

% \section{Tables}

% Use vertical lines sparingly in tables. They're unnecessary bloat. Write the code for tables in a separate tex file, and include it in when required. Also preferrably use sans serif font for tables (because of their information density) using \texttt{\\sffamily} in table definition.

% \begin{table}[!ht]
% \small
% \centering
% \sffamily
% \input{tables/table.tex}
% \caption{This is a table}
% \label{tab:table}
% \vspace{-5mm}
% \end{table}

